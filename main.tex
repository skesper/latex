\documentclass[]{book}

% Damit wir Umlaute verwenden können. Achtung! UTF-8 Encoding, nicht jeder Editor kann das.
% Besonders Windows Editoren verwenden oft das windows-1252 bzw. Latin-1252 Encoding.
\usepackage[utf8]{inputenc}
\usepackage[T1]{fontenc}
\usepackage[ngerman]{babel}

% Wird für \printindex benötigt.
\usepackage{makeidx}

% Andere Fonts
%\usepackage{cmbright}
\usepackage{fouriernc}

% AMS Mathe Bibliotheken
\usepackage{amsmath}
\usepackage{amssymb}
\usepackage{amsthm}
\usepackage{amsfonts}

% Bilder einfügen.
\usepackage{graphicx}

% Dingbats
\usepackage{pifont}

% Beispiel Fähigkeiten über Packages
\usepackage{skak}
\usepackage{yfonts}
\usepackage{color}
\usepackage{tikz}
\usetikzlibrary{calc}
\usetikzlibrary{intersections}
\usetikzlibrary{backgrounds}
\usepackage{recycle}
\usepackage{staves}
\usepackage{hieroglf}

\usepackage{pdfpages}

\makeatletter

\tikzset{%
  fancy quotes/.style={
    text width=\fq@width pt,
    align=justify,
    inner sep=1em,
    anchor=north west,
    minimum width=\textwidth,
  },
  fancy quotes width/.initial={.8\textwidth},
  fancy quotes marks/.style={
    scale=8,
    text=white,
    inner sep=0pt,
  },
  fancy quotes opening/.style={
    fancy quotes marks,
  },
  fancy quotes closing/.style={
    fancy quotes marks,
  },
  fancy quotes background/.style={
    show background rectangle,
    inner frame xsep=0pt,
    background rectangle/.style={
      fill=gray!25,
      rounded corners,
    },
  }
}

\newenvironment{fancyquotes}[1][]{%
\noindent
\tikzpicture[fancy quotes background]
\node[fancy quotes opening,anchor=north west] (fq@ul) at (0,0) {!};
\tikz@scan@one@point\pgfutil@firstofone(fq@ul.east)
\pgfmathsetmacro{\fq@width}{\textwidth - 2*\pgf@x}
\node[fancy quotes,#1] (fq@txt) at (fq@ul.north west) \bgroup}
{\egroup;
\node[overlay,fancy quotes closing,anchor=east] at (fq@txt.south east) {};
\endtikzpicture}


% Das eigentliche Dokument
\begin{document}

\frontmatter
\pagestyle{empty}

% --------------- TITLE PAGE
\begin{flushright}
{\Large Stephan Kesper}
\vskip 3cm
{\fontsize{48}{52} \selectfont \LaTeX}
\vskip 1cm
{\Large \textsl{Eine Einführung mit Best Practices}}
\vskip 5cm
\today
\vskip 2cm
\vfill
\fbox{
{\fontsize{28}{50} \textpmhg{a}} \begin{minipage}[b]{75px}
\begin{flushright}
A RAVEN\\
FRAMEWORK\\
PROJECT
\end{flushright}
\end{minipage}}
\end{flushright}
%\includegraphics{../../../License/by-nc-sa.png}
% --------------- TITLE PAGE
\newpage



\setcounter{page}{1}

\tableofcontents
\newpage

\listoffigures
\newpage

\listoftables
\newpage


\mainmatter


\chapter{Grundlagen}

\section{Was ist \LaTeX ?}

Ein klein wenig Geschichte. 1977 begann die Entwicklung von \TeX\ durch Prof. Donald E. Knuth\index{Knuth, Donald E.}. Laut Hören-Sagen war Knuth mit dem Satz seines grundlegenden Werkes "`The Art of Computer Programming"' durch seinen Verlag Ad\-di\-son-Wesley unzufrieden. Daraufhin beschloss er, sein eigenes Satz System zu entwickeln. Was als Projekt "`über den Sommer"' hin konzipiert war, dauerte insgesamt elf Jahre. 

\TeX\ besteht zunächst aus einem Satz von Programmen. Diese sind einem Compiler nicht unähnlich. Der \TeX-Compiler liest den "`Source-Code"' ein und bereitet diesen in einer vom Ausgabe-Gerät unabhängigen Weise auf (englisch Device Independent, abk. DVI)\index{DVI}. Des Weiteren gab es ein Anzeigeprogramm für DVI Dateien mit einer Druckfunktion, sowie weitere Programme mit unterschiedlichsten Zielen. Wichtig dabei ist noch das dvi2ps Tool, das Device Independent Dateien in sogenannte PostScript (PS) Dateien umwandeln konnte. PostScript ist eine Druckerbeschreibungssprache, die besonders in den achtziger und neunziger Jahren Verwendung fand. Mit Aufkommen des PDF Formats spielte PostScript aber keine ernsthafte Rolle mehr. Heute wird auch oft das pdftex und pdflatex verwendet, deren Ausgabe direkt, ohne Umweg über das DVI Format, in PDF Dateien erfolgt. 

In der Folge wurden viele Zusatztools und Erweiterungen zu \TeX\ geschrieben, sodass aus dem ursprünglich einfachen Satzsystem für wissenschaftliche Veröffentlichungen ein unglaublich mächtiges Tool geworden ist. 

Eine der wichtigsten und zentralen Erweiterungen ist das sogenannte Lamport \TeX, ein Sammlung von Makros, die durch Leslie Lamport\index{Lamport, Leslie} anfang der 1980er Jahre entwickelt und veröffentlicht wurde. Von Lamports Namen leitet sich die Vorsilbe "`La"' in \LaTeX\ ab.

\section{Wo beginnen wir?}

Aufgrund dessen, dass man mit \LaTeX\ ohne hin alles tun kann, was man in \TeX\ tun kann -- und noch viel, viel mehr, fangen wir direkt mit \LaTeX\ an. Zuerst \TeX\ zu lernen bringt keinen ernsthaften Vorteil. 

\subsection{Das minimale Dokument}

Wie in der Programmierung üblich, beginnen wir mit einem "`Hello World"'\index{Hello World}
\begin{verbatim}
\documentclass{article}

\begin{document}
Hello World.
\end{document}
\end{verbatim}
und speichern dieses in einer Datei \texttt{hello.tex}. Der Befehl zum Übersetzen
\begin{verbatim}
pdflatex hello.tex
\end{verbatim}
daraufhin erfolgt diese Ausgabe:
\footnotesize
\begin{verbatim}
This is pdfTeX, Version 3.1415926-2.4-1.40.13 (TeX Live 2012/W32TeX)
 restricted \write18 enabled.
entering extended mode
(./hello.tex
LaTeX2e <2011/06/27>
Babel <v3.8m> and hyphenation patterns for english, dumylang, nohyphenation, ge
rman-x-2012-05-30, ngerman-x-2012-05-30, afrikaans, ancientgreek, ibycus, arabi
c, armenian, basque, bulgarian, catalan, pinyin, coptic, croatian, czech, danis
h, dutch, ukenglish, usenglishmax, esperanto, estonian, ethiopic, farsi, finnis
h, french, friulan, galician, german, ngerman, swissgerman, monogreek, greek, h
ungarian, icelandic, assamese, bengali, gujarati, hindi, kannada, malayalam, ma
rathi, oriya, panjabi, tamil, telugu, indonesian, interlingua, irish, italian,
kurmanji, latin, latvian, lithuanian, mongolian, mongolianlmc, bokmal, nynorsk,
 polish, portuguese, romanian, romansh, russian, sanskrit, serbian, serbianc, s
lovak, slovenian, spanish, swedish, turkish, turkmen, ukrainian, uppersorbian,
welsh, loaded.
(d:/texlive/2012/texmf-dist/tex/latex/base/article.cls
Document Class: article 2007/10/19 v1.4h Standard LaTeX document class
(d:/texlive/2012/texmf-dist/tex/latex/base/size10.clo)) (./hello.aux) [1{d:/tex
live/2012/texmf-var/fonts/map/pdftex/updmap/pdftex.map}] (./hello.aux) )<d:/tex
live/2012/texmf-dist/fonts/type1/public/amsfonts/cm/cmr10.pfb>
Output written on hello.pdf (1 page, 11852 bytes).
Transcript written on hello.log.
\end{verbatim}
\normalsize
Diese Informationen sollten Sie nicht einschüchtern. \LaTeX\ ist ziemlich rede-freudig und informiert einen über alle wichtigen und unwichtigen Details. Gehen wir hier einmal am Beispiel die Ausgaben durch, doch das werden wir nur einmal tun. Später verwenden wir einen Editor, der uns die Arbeit der Analyse des Logfiles abnehmen wird. 

Die Ausgabe beginnt mit einer Begrüßung und der Programmversion. Danach wird \LaTeX2e aufgerufen in der Version vom 27.06.2011. Es folgt das Laden des Babel Pakets das einen darüber informiert, für welche Sprachen Silben-Trennungsinformationen vorhanden sind. Danach schließt sich ein interessanter Teil an, nämlich das Laden der Dokumentklasse, in unserem Fall die "`article.cls"'. Anschließend folgen Ausgaben über die Zwischendateien und letztlich die Angabe, dass der "`Output"' in die Datei "`hello.pdf"' geschrieben wurde. Und zwar eine Seite mit 11852 Bytes Länge. Sowie Informationen des Vorgangs in der Datei "`hello.log"' abgelegt wurden. 

Ich erspare mit die Darstellung der hello.pdf Datei. Sie besteht aus einer Din-A4 Seite mit den beiden Worten "`Hello World"'.

In Kapitel \ref{chap:doc} gehe ich darauf ein, welche Bedeutung die Dokumentklasse hat. Hier sei nur erwähnt, dass sie eine Art Schablone und/oder Korsett für unser Dokument darstellt. Die Dokumentklasse bestimmt in weiten Teilen das Aussehen des Dokuments, sofern man dies nicht wieder ändert. Und sie enthält ggf. Voreinstellungen sowie neue Befehle, die vom Hersteller der Dokumentklasse für bestimmte Zwecke erzeugt und zusammengestellt wurden. 

\section{Distribution und Installation}

Es gibt diverse \TeX-Distributionen. Eine der umfangreichsten und gleichzeitig meist-verwendeten ist die \TeX-Live Distribution der \TeX-User-Group (TUG)\index{TUG}. Unter der URL 
\begin{verbatim}
https://www.tug.org/texlive/acquire-iso.html
\end{verbatim}
bekommen Sie ein ISO Image der gerade aktuellen \TeX-Live\index{\TeX-Live} Distribution. Es gibt auch einen sogenannten "`net-installer"'. Ich habe diesen diverse Male ausprobiert, scheiterte aber immer wieder an (vermutlich) meinem Internetprovider. Bei mir schlug der Netzinstalliervorgang immer nach einigen Stunden fehl und ich konnte von Neuem beginnen. Hinzu kommt, dass der Download des ISO Files i.a. schneller ist, als für jede Datei einzeln einen Download durch den Net-Installer durchführen zu lassen. 

Das ISO File kann man entweder als Laufwerk Mounten\footnote{Hierfür braucht man eine zusätzliche Software, z.B. das kostenlose Virtual CloneDrive der Firma slysoft.com. } oder brennt sie auf ein DVD Rohling. 

In beiden Fällen erfolgt dann die Installation identisch wie mit dem Net-Installer nur mit dem Unterschied, dass die Ressourcen nicht aus dem Internet nachgeladen werden. 

Bei der Installation wird man nach einigen Informationen gefragt. Zum Beispiel Installationsort, bevorzugte Papiergröße usw. Nach der Installation hat man ein etwa drei Giga-Byte großen Ordner auf der Festplatte, soviel Platz sollte also noch mindestens auf Ihrer Festplatte frei sein.

\section{Editoren}

Auch wenn man \LaTeX\ Dokumente ohne einen "`vernünftigen"' Editor schreiben kann und die Ausführung der Befehle von Hand in einer Shell möglich ist, sollte man überlegen, ob dies eine gute Vorgehensweise ist. 

Es gibt einige \LaTeX\ Editoren und jeder von ihnen macht dem Anwender das Leben auf verschiedene Art einfacher. Denn "`let's face it"' \LaTeX\ zu schreiben ist nicht immer ein Vergnügen. Besonders dann nicht, wenn man lange Befehle schreiben muss, um etwas sehr einfaches zu erreichen. So wie ich hier 
\begin{verbatim}
\begin{equation}
a^2+b^2=c^2
\end{equation}
\end{verbatim}
schreiben muss, um die Gleichung
\begin{equation}
a^2+b^2=c^2
\end{equation}
zu setzen.

Auch wenn ich hier drei Editoren anspreche, so ist die Wahl des geeigneten Editors immer eine Frage des persönlichen Geschmacks. Manche wollen, dass der Editor möglichst schnell und leichtgewichtig ist und verzichten dafür auf Funktionalität. Andere bevorzugen eine integrierte und möglichst weitgehende Unterstützung beim Schreiben von \TeX-Makros. Die Entscheidung, welcher Editor zum Einsatz kommt, sollten Sie in Ruhe alleine treffen. Alle diese Editoren haben gewisse Vor- und Nachteile. Es spricht nichts dagegen, auch mehrere Editoren für verschiedene Aufgaben zu verwenden. Machen Sie sich selbst ein Bild!


\subsection{TeXworks}\index{TeXworks}

TeXworks wird während der Installation von \TeX-Live Distribution gleich mit installiert -- sofern man ihn nicht abgewählt hat. TeXworks ist ein zwei Fester Editor, im linken Fenster schreibt man den \TeX-Code und im rechten wird das Dokument in PDF oder DVI dargestellt. Folgende Features sind enthalten:

\begin{enumerate}
\item Freie Wahl des \TeX-Übersetzers. 
\item Syntax Highlighting
\item Schnell und schlank, d.h. schnell geladen.
\end{enumerate}

Leider kann ich nicht viel mehr über diesen Editor sagen. Er ist sicher ausreichend, um mit \LaTeX\ zu arbeiten, aber Unterstützung bekommt man nur in sehr eingeschränkter Form.

\subsection{TeXstudio und TeXmaker}\index{TeXstudio}\index{TeXmaker}

TeXstudio und TeXmaker sind relativ ähnliche Editoren mit einem vergleichbaren Funktionsumfang. Ich bin bei beiden nicht in die allerletzten Tiefen vorgedrungen, vielleicht unterscheiden sie sich doch noch irgendwo, aber auf den ersten Blick fällt es mir schwer einen markanten Unterschied festzustellen. 

Allerdings ist das TeXstudio mein bevorzugter \LaTeX-Editor. Das liegt vor allem an zwei kleinen Features: 

\begin{enumerate}
\item Wenn Sie ein längeres Dokument erzeugen, werden Sie Kapitel in eigene Dateien auslagern, um einen besseren Überblick zu bekommen (siehe Abschnitt \ref{sect:import}). Diese importierten Dateien sind keine vollständigen \LaTeX-Dokumente. Wenn man sie zu übersetzen versucht, schlägt dies fehl. TeXstudio versteht die Struktur und wird, wenn Sie innerhalb einer Kapiteldatei arbeiten, das Gesamtdokument übersetzen, nicht die Datei, an der Sie gerade arbeiten. Dies ist zwar ein kleines aber ungemein hilfreiches Feature. 
\item Das zweite ist ein Feature, das aus der Programmierung bekannt ist: Hat man ein import/include Statement oder eine Referenz auf einen Label (siehe Abschnitt \ref{sect:verweise}), so kann man mit der STRG-Taste auf diese Anweisung klicken und folgt diesen -- quasi wie einem Link -- zur importierten Datei oder der Label Definition. Das erspart viel Sucherei vor allem dann, wenn die Label Definition in einer anderen Datei liegt. 
\end{enumerate}

Im restlichen Betrieb tun sich TeXstudio und TeXmaker nicht viel, sie sind vom Umfang und von der Nutzung her sehr ähnlich. Stellt man das Oberflächendesign auf "`modern"' -- das geht bei beiden -- kann man sie kaum voneinander unterscheiden. Ich vermute, dass dies daran liegt, weil beide Editoren das Qt-Framework für ihr GUI verwenden. 




\section{Showroom}

Hier ein paar Pakete, die Einsicht in die Leistungsfähigkeit von \LaTeX\ bieten sollen.

\subsection{Skak Paket}

Schach-Enthusiasten werden sich darüber freuen. Das Skak\index{Skak-Paket} Paket bietet auf sehr einfache Weise die Möglichkeit Schachpartien zu dokumentieren. Die Eingabe folgt dabei der üblichen Nomenklatur. Das in Abbildung \ref{fig:rybov} dargestellte Schachrett wurde mit den folgenden Befehlen erzeugt:
\begin{verbatim}
\newgame
\mainline{1.e4 e5}

\mainline{2. Nf3 Nc6 3.Bb5}

\showboard
\end{verbatim}

\bigskip

\newgame
\mainline{1.e4 e5}

\mainline{2. Nf3 Nc6 3.Bb5}

\begin{figure}[h]
\centering
\showboard
\caption{Die Rybov-Variante}
\label{fig:rybov}
\end{figure}

\newpage
\subsection{Zeichnungen mit Ti\textit{k}Z}

Mit dem Ti\textit{k}Z und PGF\index{TikZ}\index{PGF} Paket hat man unter \LaTeX\ eine herausragende Möglichkeit Zeichnungen herzustellen. Das Beispiel in Abbildung \ref{fig:tikz} ist aus dem Ti\textit{k}Z Manual \cite{tikman} von Till Tantau\index{Tantau, Till} entnommen.

\begin{figure}[h]
\begin{tikzpicture}
[scale=2,line cap=round,
% Styles
axes/.style=,
important line/.style={very thick},
information text/.style={rounded corners,fill=red!10,inner sep=1ex}]
% Local definitions
\def\costhirty{0.8660256}
% Colors
\colorlet{anglecolor}{green!50!black}
\colorlet{sincolor}{red}
\colorlet{tancolor}{orange!80!black}
\colorlet{coscolor}{blue}
% The graphic
\draw[help lines,step=0.5cm] (-1.4,-1.4) grid (1.4,1.4);
\draw (0,0) circle (1cm);
\begin{scope}[axes]
\draw[->] (-1.5,0) -- (1.5,0) node[right] {$x$} coordinate(x axis);
\draw[->] (0,-1.5) -- (0,1.5) node[above] {$y$} coordinate(y axis);
\foreach \x/\xtext in {-1, -.5/-\frac{1}{2}, 1}
\draw[xshift=\x cm] (0pt,1pt) -- (0pt,-1pt) node[below,fill=white] {$\xtext$};
\foreach \y/\ytext in {-1, -.5/-\frac{1}{2}, .5/\frac{1}{2}, 1}
\draw[yshift=\y cm] (1pt,0pt) -- (-1pt,0pt) node[left,fill=white] {$\ytext$};
\end{scope}
\filldraw[fill=green!20,draw=anglecolor] (0,0) -- (3mm,0pt) arc(0:30:3mm);
\draw (15:2mm) node[anglecolor] {$\alpha$};
\draw[important line,sincolor]
(30:1cm) -- node[left=1pt,fill=white] {$\sin \alpha$} (30:1cm |- x axis);
\draw[important line,coscolor]
(30:1cm |- x axis) -- node[below=2pt,fill=white] {$\cos \alpha$} (0,0);
\path [name path=upward line] (1,0) -- (1,1);
\path [name path=sloped line] (0,0) -- (30:1.5cm);
\draw [name intersections={of=upward line and sloped line, by=t}]
[very thick,orange] (1,0) -- node [right=1pt,fill=white]
{$\displaystyle \tan \alpha \color{black}=
\frac{{\color{red}\sin \alpha}}{\color{blue}\cos \alpha}$} (t);
\draw (0,0) -- (t);
\draw[xshift=2.2cm]
node[right,text width=6cm,information text]
{
The {\color{anglecolor} angle $\alpha$} is $30^\circ$ in the
example ($\pi/6$ in radians). The {\color{sincolor}sine of
$\alpha$}, which is the height of the red line, is
\[
{\color{sincolor} \sin \alpha} = 1/2.
\]
By the Theorem of Pythagoras ...
};
\end{tikzpicture}
\caption{Beispiel Zeichnung mit dem TikZ Package}
\label{fig:tikz}
\end{figure}


\subsection{Sonstiges}

\begin{figure}[h]
\centering
\begin{minipage}[b]{0.5\textwidth}
\frakfamily
\yinipar{L}orem ipsum dolor sit amet, consetetur sadipscing elitr, sed diam nonumy eirmod tempor invidunt ut labore et dolore magna aliquyam erat, sed diam voluptua. At vero eos et accusam et justo duo dolores et ea rebum. Stet clita kasd gubergren, no sea takimata sanctus est Lorem ipsum dolor sit amet. Lorem ipsum dolor sit amet, consetetur sadipscing elitr, sed diam nonumy eirmod tempor invidunt ut labore et dolore magna aliquyam erat, sed diam voluptua. At vero eos et accusam et justo duo dolores et ea rebum. Stet clita kasd gubergren, no sea takimata sanctus est Lorem ipsum dolor sit amet.
\end{minipage}
\caption{Altertümliche anmutende Texte mit dem \texttt{yfonts} Paket}
\label{fig:lorem}
\end{figure}

\begin{figure}[h]
\centering
\recycle
\Recycle
\RECYCLE
\caption{Aus dem \texttt{recycle} Paket}
\label{fig:recycle}
\end{figure}

\begin{figure}[h]
\centering
{\fontsize{64}{66} \staveXLV \staveXLIII \staveXX}
\caption{\texttt{staves} Paket, Bedeutung nachzulesen auf der Website des Museum of Icelandic Sorcery and Witchcraft}
\label{fig:staves}
\end{figure}

\begin{figure}[h]
\centering
{\fontsize{48}{50} \textpmhg{Rekin}}
\caption{\texttt{hieroglf} Paket zur Darstellung ägyptischer Hieroglyphen.}
\label{fig:hiero}
\end{figure}

Man kann wohl ohne Übertreibung sagen, dass \LaTeX\ eines der vielseitigsten Satzprogramme ist, vor allem auch deswegen, weil viele \LaTeX-Enthusiasten auf der ganzen Welt unermüdlich an der Weiterentwicklung der Pakete und damit den Fähigkeiten von \LaTeX\ arbeiten. Ob es Fraktur-Fonts wie in Abbildung \ref{fig:lorem} sind, oder Sondersymbole wir in Abbildung \ref{fig:recycle}, Symbole der Isländischen Folklore in Abbildung \ref{fig:staves} oder auch ägyptische Hieroglyphen wie in Abbildung \ref{fig:hiero}, wenn man etwas sucht, findet man für fast alle Anwendungsgebiete ein \LaTeX\ Paket -- solange dieses Gebiet etwas mit der Veröffentlichung von Dokumenten zu tun hat. 

\section{Zusammenfassung}

Für die meisten (selbst so manch exotische) Arten von Veröffentlichungen bietet \LaTeX\ ein Paket zur Unterstützung an. Auch wenn die meisten dieser Pakete bereits in der \TeX-Live Distribution enthalten sind und man eigentlich nur im Installationsordner nachschauen müsste, ist eine kurze Google Suche nach existierenden Paketen immer sinnvoll. Denn man findet oft auch gleichzeitig gute und vielseitige Beispiele zum Einsatz des Pakets. 

Versucht man ein Paket mit dem \texttt{$\backslash$usepackage} Befehl zu verwenden und schlägt dies fehl, dann muss man dieses Paket separat installieren. Dies wird in Kapitel \ref{chap:instpack} noch erklärt. Alle, die bereits installiert sind, können nach der Deklaration über den usepackage Befehl verwendet werden.





\chapter{Dokumente und Klassen}\label{chap:doc}

Wie bereits gesagt besitzen \LaTeX\ Dokumente immer eine Dokument-Klasse\index{Dokument-Klasse}. Diese bestimmt in erster Linie das Aussehen des finalen Dokuments. Wie die Menge an Packages vermuten lässt, gibt es auch eine recht unübersichtliche Anzahl von Dokument-Klassen. Wir werden uns um zwei \LaTeX\ Hauptklassen kümmern, zum einen die \texttt{article}-Klasse sowie die \texttt{book}-Klasse.

Später werden wir noch die \texttt{amsbook}-Klasse und \texttt{tufte}-Klassen kennen lernen. Die KOMA-Skript Klassen hätten sicher einen Platz hier verdient, jedoch ist ihre Funktionalität so umfangreich, dass wir an dieser Stelle nur auf ihre Dokumentation verweisen können mit dem Hinweis, dass es sich lohnt damit auseinander zu setzen.
\begin{verbatim}
http://www.komascript.de/
\end{verbatim}\index{komaspcript}
Für den Anfang -- zum Beispiel im Kontext einer Diplom-, Master-Arbeit oder einer Dissertation -- werden wir uns hier auf die Grundlagen beschränken.

Ein \LaTeX\ Dokument hat folgende Struktur: 
\begin{verbatim}
\documentclass{...}
% Präambel
\begin{document}
% Inhalt...
\end{document}
\end{verbatim}
Der erste Befehl muss immer die Angabe der Dokumentklasse sein. Der Bereich zwischen Dokumentklasse und dem Befehl
\begin{verbatim}
\begin{document}
\end{verbatim}
wird \textbf{Präambel}\index{Präambel} genannt. Dort werden alle Einstellungen untergebracht. Alles, was dort steht, hat nur indirekt Einfluss auf das Dokument. Es darf dort kein Text stehen! 

Zwischen \texttt{begin\{document\}} und \texttt{end\{document\}} steht der eigentliche Inhalt. Alles was dort steht, wird als Text interpretiert und in die Ausgabe des Dokuments mitaufgenommen. Lediglich die Befehle, die grundsätzlich mit einem Backslash $\backslash$ beginnen, wandelt \LaTeX\ nicht in Text um, sondern führt diese aus.

Überblick über die Sonderzeichen, die in \LaTeX\ eine Bedeutung haben:

\begin{description}
\item[$\backslash$] Beginnt einen Befehl.
\item[\$] Beginnt und beendet den inline Mathematikmodus. Dieser wird in Abschnitt \ref{sec:inline} näher beschrieben.
\item[\%] Beginnt einen Kommentar. Alles, was in einer Zeile hinter einem \%-Zeichen folgt, wird als Kommentar interpretierung und von \LaTeX\ ignoriert.
\item[\&] Ist ein sogenannter Tab-Character. In Tabellen oder Matrizen wird dieser verwendet. Die Nutzung wird in den Kapiteln \ref{chap:table} und \ref{chap:formel} erklärt.
\end{description}

Bis auf das Backslash kann jedes dieser Zeichen "`Escaped"'\index{Escaped} werden, das bedeutet, man zeigt \LaTeX\ an, dass man das Zeichen selbst benutzen will und nicht an seine Bedeutung für \LaTeX\ interessiert ist. Dies macht man mit dem Backslash. Also schreibt man das solche Zeichen einfach als 
\begin{verbatim}
\% \& \$
\end{verbatim}
leider gilt das nicht für das Backslash selbst. Denn das Doppelbackslash hat wiederum eine eigene Bedeutung als Abkürzung für die Beendigung der aktuellen Zeile. Wenn man im Text also \\
eine neue Zeile anfangen will, \\
ohne einen neuen Absatz zu erzeugen, \\
verwendet man das doppelte Backslash.

Der Befehl 
\begin{verbatim}
\end{document}
\end{verbatim}
beendet (suggestiver Weise) das Dokument. Alles, was dahinter kommt, wird ignoriert. 


\section{Absätze}\index{Absatz}

\LaTeX\ interpretiert einen einzigen Newline-Character, wie er z.B. durch Betätigung der Return Taste erzeugt wird, nicht als ernsthaften Versuch einen Absatz zu beenden. Man muss eine echte Leerzeile erzeugen, damit ein Absatz erzeugt wird. Also
\begin{verbatim}
Erste Zeile,
hier wird noch kein neuer Absatz erzeugt. ...

Aber dafür hier.
\end{verbatim}
Wird von \LaTeX\ zu:

\bigskip

\small
Erste Zeile,
hier wird noch kein neuer Absatz erzeugt. Hier schreibe ich noch die Zeile voll, damit der erste Umbruch erreicht wird.

Aber dafür hier.
\normalfont

\bigskip

Ein Absatz beginnt mit dem Einzug der ersten Zeile. Die Breite dieses Einzuges wird durch den Parameter \index{parindent}
\begin{verbatim}
\parindent
\end{verbatim}
bestimmt. Die erste Zeile eines Kapitels oder einer (sub-)section wird ohne Einzug dargestellt. Möchte man für einen Absatz den Einzug verhindern, verwendet man den 
\begin{verbatim}
\noindent
\end{verbatim}
Befehl. Möchte man ganz auf Einzüge verzichten, kann man mit 
\begin{verbatim}
\setlength{\parindent}{0pt}
\end{verbatim}
den \texttt{parindent}-Wert auf 0 setzen. Dies kann in der Präambel für das gesamte Dokument geschehen, oder auch zwischendurch, wenn man nur für einen gewissen Teil den Wert verändern möchte. Eine Änderung wirkt sich auf alle folgenden Absätze aus. Möchte man die Absatz-Einzüge wieder verwenden, muss der \texttt{parindent}-Wert auf den vorherigen Wert gesetzt werden. Der Default Wert liegt bei 15pt\footnote{pt Maßeinheit "`Punkt"'.}. Eine Dokumentklasse kann allerdings eigene Einzugsbreiten definieren. Darauf sollte man achten, wenn man den \texttt{parindent}-Wert ändert.

\subsection{Silbentrennung}\index{Silbentrennung}

Die Trennung von Wörtern ist eigentlich eine sprachspezifische Vorgehensweise und wäre somit im folgenden Kapitel besser aufgehoben. Auf der anderen Seite ist die Hauptaufgabe der Silbentrennung, die Formatierung eines Absatzes zu erleichtern. 

TODO

\section{Sprache}

Ein \LaTeX\ Dokument besitzt eine Spracheinstellung, welche auf "`english"' voreingestellt ist. Sie bestimmt in erster Linie, welche Silbentrennung verwendet wird. Im Falle einer \texttt{book}-Klasse gibt es aber auch noch automatische Texte, wie zum Beispiel die Überschriften der Inhalts-, Abbildungs- und Tabellenverzeichnisse, des Weiteren erscheint über jedem Kapitel neben der vorgegebenen Überschrift auch noch das Wort "`Kapitel"' mit Nummer. Diese Texte können für die entsprechenden Sprachen von der Dokumentklasse automatisch angepasst werden. 

\index{Spracheinstellung} \index{Deutsch}
Zur Definition von Deutsch als Dokumentsprache, muss in der Präambel die folgende Einstellung gemacht werden:
\begin{verbatim}
\usepackage[ngerman]{babel}
\end{verbatim}\index{babel-Paket}
Das \texttt{babel}-Paket wird daraufhin Deutsch als Sprache angeben, sodass alle Silbentrennungen nach den deutschen Regeln geschehen. Zusätzlich wird die \texttt{book}-Klasse die automatischen Texte auf deutsch umstellen. Der Parameter \texttt{ngerman} bezeichnet die Silbentrennung nach der neuen Rechtschreibung\footnote{Rechtschreibreform von 1996}, während der (ebenfalls noch anwählbare) \texttt{german} Parameter die alte Rechtschreibung verwenden würde. 

Weitere Auswirkungen hat die Spracheinstellung nicht (im Besonderen keine Rechtschreibprüfung -- dies realisiert der Editor). Zu beachten ist, dass eine Umstellung der Sprache nicht bedeutet, dass auch im Eingabetext Umlaute usw. verwendet werden können. Hierfür ist das im Folgenden beschriebene Input Encoding verantwortlich.

\subsection{Input-Encoding}\index{Input-Encoding}

Früher war es nicht möglich, in \TeX\ Umlaute zu verwenden. Damals musste man z.B. ein ü als \texttt{$\backslash$"\,u} eingeben. Was zu der ohnehin schon umständlichen Eingabemethode hinzukam. Extrem erleichtert wurde die Situation dadurch, dass man irgendwann ein "`Input-Encoding"' einführte. Dieses bestimmt -- salopp gesprochen -- ein Encoding derjenigen Zeichen, die der Anwender auf seiner Tastatur tippt. 

Für deutsche Texte würde es ausreichen, ein einfaches Encoding wie ISO-8859-15 (inkl. Euro Zeichen) oder windows-1252 zu verwenden. Ich persönlich lasse mir aber gerne alle Freiheiten offen, sodass ich das UTF-8 Encoding bevorzuge. Das hat allerdings zur Folge, dass man beim Erzeugen von Dokumenten (\LaTeX-Dateien) unter Umständen nicht einfach irgendeinen Editor verwenden kann, sondern einen, der UTF-8 Dokumente erzeugen kann. Ein \LaTeX-Editor wird dies immer tun. Aber zum Beispiel das Windows Notepad nur auf Anforderung. Verwendet man das Notepad, so muss man beim Speichern einer Datei angeben, dass "`UTF-8"' als Encoding verwendet wird. Es ist möglich, aber man muss explizit daran denken.

Ist das Input-Encoding auf ein bestimmtes Encoding eingestellt, werden Dateien, die ein anderes Encoding haben entweder seltsame Zeichen produzieren, oder gar nicht lesbar sein. 

Das Input-Encoding wird über den folgenden Befehl auf UTF-8 umgestellt:\index{inputenc-Paket}
\begin{verbatim}
\usepackage[utf8]{inputenc}
\end{verbatim}

\subsection{Font-Encoding}\index{Font-Encoding}

Das sogenannte Font-Encoding ist das dritte Standbein der Spracheinstellungen. Es würde an dieser Stelle etwas zu weit führen, es in aller Tiefe zu diskutieren. Als kurzer Überblick sei erwähnt, dass das \TeX\ am Anfang nur auf die englische Sprache abzielte und daher einen relativ beschränkten Satz an Sonderzeichen (im normalen Text, dies betrifft nicht den Formelsatz) unterstützte. 

Als \TeX\ zunehmend internationalisierter wurde, war der Druck Sonderzeichen einzuführen immer größer. Aus den ursprünglich zur Verfügung stehenden 128 Glyphen Sätzen wurden 256 Glyphen Sätze. Die 128 Gylphen Sätze wurden mit einem "`O"' bezeichnet, für "`Original"' oder halt eben "`old"'. Tabelle \ref{tab:encodings} zeigt eine Übersicht über die wichtigsten (nicht alle) Font-Encodings. Für deutsche Texte ist eigentlich nur T1 relevant und von den O... Encodings sollte man die Finger lassen.

\begin{table}[h]
\centering
\begin{tabular}{r|c|l}
\hline
\textbf{Name} & \textbf{Glyphen} & \textbf{Beschreibung} \\
\hline
OT1 & 128 & Font Encoding von D. E. Knuth \\
OT2 & 128 & kyrillisches Font Encoding \\
OT3 & 128 & Font Encoding für phonetische Anwendungen \\
OT4 & 128 & Font Encoding für polnische Sonderzeichen \\
OT6 & 128 & Font Encoding für Armenische Sprache \\
T1 & 256 & Font Encoding für europäische Sprachen \\
T2A, T2B, T2C & 256 & Font Encodings für kyrillische Sprachen \\
T3 & 256 & Font Encoding für phonetische Anwendungen \\
T4 & 256 & Font Encoding für afrikanische Sprachen \\
T5 & 256 & Font Encoding für Vietnamesisch \\
\hline
\end{tabular}\\[3mm]
\caption{Übersicht Font Encodings} \label{tab:encodings}
\end{table}


\subsection{Zusammenfassung}

Für eine Arbeit in deutscher Sprache, unter Verwendung direkter Umlaute und einem korrekten Font-Encoding, sollten die Einstellungen wie folgt geschehen:
\begin{verbatim}
\usepackage[utf8]{inputenc}
\usepackage[T1]{fontenc}
\usepackage[ngerman]{babel}
\end{verbatim}

Solange man in deutsch arbeitet, können diese drei Angaben immer in der Präambel auftauchen, damit steht man auf der sicheren Seite. Zu beachten ist, dass man bei der Einstellung eines UTF-8 Input-Encodings sich auch eines UTF-8 fähigen Editors versichern muss. 

\section{Struktur}\index{Struktur}

\LaTeX\ Dokumente besitzen eine Struktur, die auf Teilen, Kapiteln und Unterabschnitten beruht. Tabelle \ref{tab:structure} stellt die Ordnungsebenen der jeweiligen Struktur-Elemente dar:

\index{Kapitel} \index{Section} \index{Subsection}
\begin{table}[h]
\centering
\begin{tabular}{l|l|l}
Ebene -1 & \texttt{$\backslash$part} & Teil \\
Ebene 0 & \texttt{$\backslash$chapter} & Kapitel \\
Ebene 1 & \texttt{$\backslash$section} & Abschnitt \\
Ebene 2 & \texttt{$\backslash$subsection} & Unterabschnitt \\
Ebene 3 & \texttt{$\backslash$subsubsection} & Unter-Unterabschnitt \\
Ebene 4 & \texttt{$\backslash$paragraph} & Absatz \\
Ebene 5 & \texttt{$\backslash$subparagraph} & Unter-Absatz \\
\end{tabular}
\caption{Struktur Elemente in \LaTeX}
\label{tab:structure}
\end{table}

Die verschiedenen Dokument-Klassen unterstützen nicht alle Struktur-E\-le\-men\-te. Die \texttt{article}-Klasse beginnt erst ab Ebene 1. Das bedeutet, wenn Sie ein Dokument mit Klasse \texttt{article} erzeugen, führt die Verwendung der Befehle \texttt{$\backslash$part} und \texttt{$\backslash$chapter} zu Fehlern. Des Weiteren ist die Nummerierung von \texttt{section} einstellig (Beispiel: 1. Titel), während sie innerhalb von \texttt{book} Dokumenten zweistellig ist (Beispiel: 1.4 Titel) und ein umschließendes Kapitel bedingt. 

Es gibt noch weitere Unterscheidungen, die wir in den folgenden Abschnitten erklären wollen.

\section{Die Haupt-Klassen}

Hier werde ich nur auf die \texttt{article}- und \texttt{book}-Klassen eingehen. Die restlichen Hauptklassen \texttt{report}, \texttt{letter} und \texttt{beamer} werden nur erwähnt.

\subsection{\texttt{article}-Klasse}\index{article-Klasse}

Ein Artikel ist -- im Gegensatz zu einem Buch -- ein Dokument mit einem relativ überschaubaren Umfang. Artikel sind eher Facharbeiten, Veröffentlichungen in einer Zeitschrift oder komprimierte Schriften zur Wissensvermittlung. 

Ein wissenschaftlicher Artikel besteht aus einer Überschrift mit Autorenangaben, einem Abstract\footnote{Ein Abstract ist eine Zusammenfassung oder auch überblicksartige Darstellung des Inhalts des Dokuments. Es ist oft auf Englisch verfasst, auch wenn die Dokumentsprache eine andere ist. Der Abstract wird schmaler gesetzt als der eigentliche Text im Dokument sowie meist auch in einem kleineren Font.} und diversen Abschnitten, die ohne nennenswerte Abstände hintereinander im Dokument erscheinen. Ein Beispiel für einen Artikel können Sie in Abbildung \ref{fig:article} finden. 

\begin{figure}[p]
\centering
\frame{\includegraphics[width=\textwidth]{bsp/article.pdf}}
\caption{Beispiel eines Artikels}
\label{fig:article}
\end{figure}

Titel und Autor werden über die Makros \texttt{title} und \texttt{author} gesetzt. Der Befehl \texttt{maketitle} fügt diese dann zusammen zu einem Artikel-Kopf, bestehend aus Titel, Autor und dem Datum. Wenn man das Datum selbst bestimmen möchte, kann man dies über das Makro \texttt{date} setzen. 
\begin{verbatim}
\date{2. Januar 1877}
\end{verbatim}
Per Default wird immer der aktuelle Tag als Datum gesetzt. Möchte man selbst den heutigen Tag im Text verwenden, kann man dies mit dem \texttt{today} Makro. Der \LaTeX\ Code für diesen Artikel sieht folgendermaßen aus:\index{today-Befehl}
\footnotesize
\begin{verbatim}
\documentclass[]{article}

\title{Ein wissenschaftlicher Artikel}
\author{Stephan Kesper}

\begin{document}
\maketitle

\begin{abstract}
Lorem ipsum dol...
\end{abstract}

\section{Der erste Abschnitt}
Lorem ipsum dolor sit amet, ...

\section{Der zweite Abschnitt}
Lorem ipsum dolor ...

\end{document}
\end{verbatim}
\normalsize

In Artikeln ist es unüblich ein Inhaltsverzeichnis zu verwenden. Nichts-des\-to-trotz kann man es erzeugen, indem man an der Stelle, wo es dargestellt werden soll den Befehl\index{Inhaltsverzeichnis}
\begin{verbatim}
\tableofcontents
\end{verbatim}
angibt. Zum Inhaltsverzeichnis ist noch einiges zu sagen, was ich in Abschnitt \ref{chap:content} tun werde.


\subsection{\texttt{book}-Klasse}

Wenn man vor hat, einen längeren Text zu schreiben, gegebenenfalls mehrere hundert Seiten lang, so sollte man überlegen, eine Buch Klasse zu verwenden. Die einfachste unter diesen ist die \texttt{book}-Klasse.

In Büchern kann man die \texttt{part} und \texttt{chapter} Struktur-Befehle verwenden. Ein Part ist ein "`Teil"' eines Buches. Wenn Sie nicht wissen, ob und wo Sie Teile verwenden sollten, tun Sie es lieber nicht.

Die Unterteilung von Büchern in "`Parts"' geschieht meistens dann, wenn das Buch verschiedene Themenbereiche abdeckt, die einer Trennung bedürfen. Kapitel strukturieren Inhalte, aber trennen sie nicht. Letztlich bestimmt der Autor, ob er eine Einteilung seines Buches haben möchte. Trotzdem sollte er mit der Einteilung in "`Parts"' eher sparsam umgehen.

Die \texttt{book}-Klasse bietet nicht nur Inhalts-, sondern auch Tabellen- und Abbildungsverzeichnisse. Und die Kapitel beginnen -- sofern der Autor das nicht verändert -- auf der nächsten ungeraden Seite. Gegebenenfalls liegt zwischen dem letzten Text und dem neuen Kapitel eine gerade Seite, die einfach leer gelassen wird. Sections und Sub-Sections sind ähnlich wie in der \texttt{article}-Klasse zu verwenden, lediglich steht bei ihrer Nummerierung die Kapitelnummer davor. 

Man muss beachten, dass die \texttt{book}-Klasse davon ausgeht, dass der Text beidseitig gedruckt wird. Beginnend mit der ersten Seite mit Seitenzahl "`1"'. Daher ist die rechte Seite immer die ungerade Seite. 

\subsubsection{Haupt- und Nebenbereiche}

Die \texttt{book}-Klasse hat zusätzlich zur Struktur-Einteilung noch eine Einteilung in front-, main- und backmatter-Bereiche. Diese Bereiche trennen zum Beispiel die Inhalts- und Abbildungsverzeichnisse von den Hauptteilen, sowie auch von den Anhängen, Literaturverzeichnis und Index. Ein Beispiel:

\footnotesize
\begin{verbatim}
\documentclass{book}

\author{Stephan Kesper}
\title{Ein Beispiel Buch}

\begin{document}
\frontmatter
\maketitle

\tableofcontents

\mainmatter
\chapter{Erstes Kapitel}
Lorem ipsum dolor sit amet, consetetur sadipscing elitr...

\backmatter
\end{document}
\end{verbatim}
\normalsize
ergibt sechs Seiten:
\begin{description}
\item[i] Die Titelseite
\item[ii] Leere gerade Seite
\item[iii] Inhaltsverzeichnis
\item[iv] Leere gerade Seite
\item[1] Erste Inhaltsseite
\item[2] Zweite Inhaltsseite
\end{description}

\includepdf[frame,nup=2x2,pages={-}]{bsp/book.pdf}

Besondere Beachtung findet hierbei, dass der Hauptteil (mainmatter) eine eigene Nummerierung besitzt. Die Seiten des Anfangsteils (frontmatter) zählen hierbei nicht. Die frontmatter Nummerierung ist ohnehin mit römischen Zahlen realisiert, damit der Leser diese nicht mit der Nummerierung des Hauptteils verwechselt. 
\index{frontmatter} \index{mainmatter} \index{backmatter}

Das führt dazu, dass das erste Kapitel immer auf Seite 1 beginnt. Und dies auch so im Inhaltsverzeichnis dargestellt wird, auch wenn mit Inhalts-, Ab\-bil\-dungs- und Tabellenverzeichnis noch einiges an Seiten vor dem ersten Kapitel liegt. 

\begin{fancyquotes}
Die front-, main- und backmatter Befehle funktionieren nicht in der Artikel Klasse! Das ist einer der großen Unterschiede zwischen den \texttt{book}- und \texttt{article}-Klassen. Falls Sie wert auf eine konsistente Nummerierung Ihrer Bereiche legen, sollten Sie auf jeden Fall eine \texttt{book}-Klasse verwenden.
\end{fancyquotes}

\ding{229} Die Universität Koblenz (wie andere Universitäten auch) bietet Doku\-ment-Klassen für Diplom- und Magisterarbeiten zum Download an. Diese basieren auf der Article Klasse. Das bedeutet, dass die front- und backmatter Bereiche nicht verwendet werden können und so das Inhaltsverzeichnis (und ggf. weitere) in die Seitennummerierung des Dokuments mitaufgenommen würde. Um zu verhindern, dass die Nummerierung der Titelseiten die folgenden Seiten verschiebt, wird mit zwei Zählern gearbeitet. Im vorderen Bereich wird eine römische Nummerierung verwendet, während der Rest des Dokuments eine arabische Nummerierung erhält.


\subsection{\texttt{report}-Klasse}\index{report-Klasse}

Ein Report ist -- wie ein Artikel -- eher für kürzere Texte gedacht. Er hat ein \texttt{abstract}-Environment, kann allerdings Kapitel enthalten, was der Hauptunterschied zum Artikel ist. Der Abstract-Abschnitt wird auf einer eigenen Seite angezeigt. Die Kapitel beginnen auf einer neuen Seite, aber nicht notwendigerweise auf einer ungeraden Seite. 


\subsection{\texttt{letter}-Klasse}\index{letter-Klasse}

Wie der Name schon sagt, formatiert diese Klasse \LaTeX\ in einer Brief Form. Persönlich bin ich der Meinung, dass die meisten Brief Klassen von \LaTeX\ entweder so einfach sind, dass sich der Aufwand mit \LaTeX\ nicht lohnt, oder so seltsam aussehen, dass man sie nicht verwenden möchte.

\subsection{\texttt{beamer}-Klasse}\index{beamer-Klasse}

Für Präsentationen dient die \texttt{beamer}-Klasse. Speziell für wissenschaftliche Präsentationen kann es nützlich sein, \LaTeX\ zu verwenden, da natürlich der gesamte Umfang des mathematischen Formelsatzes zur Verfügung steht. Das erzeugte PDF kann mit einem PDF Betrachter im Vollbildmodus genau wie eine Powerpoint Präsentation genutzt werden. 


\section{Inhaltsverzeichnis}\label{chap:content}

\index{Inhaltsverzeichnis}
Das Inhaltsverzeichnis wird von \LaTeX\ automatisch erzeugt, wenn der Befehl 
\begin{verbatim}
\tableofcontents
\end{verbatim}
gefunden wurde. Das Inhaltsverzeichnis ist eine \LaTeX\ Datei, die den Namen Ihrer Datei hat aber die Endung "`.toc"'\footnote{toc = table of content} besitzt. Diese toc-Datei wird genau an der Stelle in Ihr Dokument eingefügt, wo der \texttt{tableofcontents} Befehl steht. In der \texttt{book}-Klasse bekommt das Inhaltsverzeichnis eine eigene Seite. Sehen wir uns einen Teil des Inhaltsverzeichnis dieses Buches an:

\footnotesize
\begin{verbatim}
\select@language {ngerman}
\contentsline {chapter}{\numberline {1}Grundlagen}{1}
\contentsline {section}{\numberline {1.1}Was ist \textlatin {\LaTeX }?}{1}
\contentsline {section}{\numberline {1.2}Wo beginnen wir?}{2}
\contentsline {subsection}{\numberline {1.2.1}Das minimale Dokument}{2}
...
\end{verbatim}
\normalsize

\LaTeX\ schreibt, während der Abarbeitung des Dokuments, einfach die Überschriften der Kapitel, Sections und Subsections in eine Datei. Und zwar mit den zu diesem Zeitpunkt gültigen Seitenzahlen und setzt den Befehl \texttt{contentsline} davor. Das ist -- wie Sie sich vorstellen können -- ein relativ naives Vorgehen. Nehmen wir an, wir hätten bereits ein längeres Dokument geschrieben und entschließen uns dann erst ein Inhaltsverzeichnis anzulegen. Wir setzen den Table of Contents Befehl an den Anfang des Codes und sehen uns an, was dabei heraus kommt. Im ersten Durchlauf gar nichts. Denn die toc-Datei war gar nicht existent und somit steht an der Stelle, wo das Inhaltsverzeichnis stehen sollte, gar nichts. Das bedeutet, nach der ersten Ausführung besitzt Ihr Dokument zwar kein Inhaltsverzeichnis, jedoch ist die toc-Datei gefüllt. Starten wir einen erneuten Durchlauf, so erscheint ein Inhaltsverzeichnis mit der zu diesem Zeitpunkt existenten toc-Datei. Das ist allerdings die toc-Datei aus dem vorherigen Lauf, weil das Inhaltsverzeichnis  \textbf{VOR} den Kapitel und Section Angaben steht (wie das für Inhaltsverzeichnisse üblicherweise der Fall ist). Deshalb stehen da noch die Seitenzahlen aus dem vorherigen Lauf. Zu diesem Zeitpunkt war aber noch kein Inhaltsverzeichnis da, also sind die Seitenzahlen verschoben, so als gäbe es kein Inhaltsverzeichnis.

Erst mit dem dritten Lauf hat das Inhaltsverzeichnis die korrekten Seitenzahlen, weil sich das Inhaltsverzeichnis auf seine volle Länge erweitert und sämtliche nachfolgenden Seiten entsprechend weiter gerückt hat.

Ich kann es nur noch einmal sagen: Diese Vorgehensweise ist naiv. Heute würde man erwarten, dass das Programm die Seitenzahlen im ersten Lauf in korrekter Weise darstellen kann. Doch das ist kein triviales Problem, denn die Länge des Inhaltsverzeichnis hat Einfluss auf die Position der Kapitel- und Abschnittsüberschriften. Aber es ist nicht unlösbar! 

Man kann nur mutmaßen, warum Prof. Knuth sich nicht die Mühe machte, eine sofortige Berechnung der Seitenzahlen zu implementieren. Auf der anderen Seite kann argumentiert werden, dass die mehrfache Ausführung des \LaTeX-Compilers (besonders bei heutigen Computern) so wenig Zeit in Anspruch nimmt, dass es keine Rolle spielt, ob \LaTeX\ ein, zwei oder dreimal ausgeführt wird. Man muss es nur Bedenken und lieber \LaTeX\ einmal mehr ausführen.

Die Art, wie das Inhaltsverzeichnis in \LaTeX\ aufgebaut wird, ist symptomatisch für alle anderen Übersichten. Der Index, Tabellen- und Abbildungsverzeichnisse werden in der selben Art erzeugt. Beim Index kommt sogar noch hinzu, dass man das Tool \texttt{makeindex} zwischen dem ersten und zweiten \LaTeX\ Aufruf starten muss. Doch dazu später mehr. 

Sie sollten für sich im Kopf behalten, dass es immer sinnvoll ist, wenn man mit \LaTeX\ arbeitet, es mehrfach zu starten. Wenn Sie die finale PDF-Version Ihrer Arbeit erzeugen, sollten Sie in der folgenden Reihenfolge vorgehen: (Nehmen wir an, Ihre Hauptdatei wäre master.tex)

\begin{enumerate}
\item \texttt{pdflatex master.tex} starten. 
\item \texttt{bibtex master.aux} starten.
\item \texttt{makeindex master.idx} starten.
\item Noch zwei Mal \texttt{pdflatex master.tex} starten. 
\end{enumerate}
Wenn Sie diesem Schema folgen, kann eigentlich nichts mehr schief gehen. Die Informationen von Kapiteln, Sections, Abbildungen und Tabellen, internen sowie Literatur Referenzen sollten nach dem letzten Start alle korrekt übernommen worden sein und auf die tatsächlichen Seiten-, Kapitel- und Formelnummern verweisen.

\ding{229} Es kann sinnvoll sein, sich dafür ein Skript zu schreiben. 

Sollten Sie mit dem TeXstudio arbeiten, reicht es, den Vorgang mit F1 zu starten. Ihre Literaturreferenzen werden automatisch erzeugt. \texttt{makeindex} können Sie direkt mit F12 aufrufen. Und dann noch zweimal F1. Also:
\begin{verbatim}
F1, F12, F1, F1
\end{verbatim}

\section{Environments}\index{Environment}

In \LaTeX\ wird der Satz von Text und Formeln von sogenannten "`Environments"' (engl. für Umgebung) bestimmt. Eines der wichtigsten Environments ist das für den Formelsatz. Aufgrund der Wichtigkeit hat es ein eigenes Kapitel bekommen, nämlich Kapitel \ref{chap:formel}.

Environments sind örtlich (im Sinne einer Textstelle) begrenzt von den Befehlen 
\begin{verbatim}
\begin{name}
...
\end{name}
\end{verbatim}
wobei \texttt{name} ein Platzhalter für den Namen des entsprechenden Environments ist. Wir werden viele Environments kennenlernen, die für die verschiedenen Formatierungen verwendet werden. Dabei wird immer nur der Name des Environments erwähnt und dabei implizit vorausgesetzt, dass Sie dieses Environment mit den begin- und end-Befehlen starten und enden lassen. Hier einige Beispiele für Environments:

\subsection{\texttt{quote} -- Zitate}\index{Zitat} \index{quote-Umgebung}

Längere Zitate sollten immer in einer Weise vom Fließtext unterschieden werden, damit nicht aus versehen dem Autor der Text zugeschrieben wird. Hierfür gibt es das \texttt{quote}-Environment.

\begin{quote}
Mag das Geld auch den Charakter des bloß Nützlichen haben, so hat es dennoch eine gewisse Ähnlichkeit mit dem Glück, weil es auch den Charakter des Allumfassenden besitzt, da ja dem Gelde alles untertan ist.
\begin{flushright}
\textsl{Thomas von Aquin (1225-1274)}
\end{flushright}
\end{quote}
Dieses Zitat würde in folgender Weise geschrieben werden:
\begin{verbatim}
\begin{quote}
Mag das Geld auch den Charakter des bloß Nützlichen haben, so 
hat es dennoch eine gewisse Ähnlichkeit mit dem Glück, weil es 
auch den Charakter des Allumfassenden besitzt, da ja dem Gelde 
alles untertan ist.
\begin{flushright}
\textsl{Thomas von Aquin (1225-1274)}
\end{flushright}
\end{quote}
\end{verbatim}
Man beachte, dass gleich zwei Environments verwendet wurden, das \texttt{quote}- und das \texttt{flushright}-Environment. Letzteres setzt den Text rechtsbündig. Hätte man dieses nicht innerhalb des \texttt{quote}-Environments verwendet, wäre es bis zum eigentlichen Seiten-Rand verschoben worden, wie im folgenden Beispiel:

\begin{quote}
Mag das Geld auch den Charakter des bloß Nützlichen haben, so 
hat es dennoch eine gewisse Ähnlichkeit mit dem Glück, weil es 
auch den Charakter des Allumfassenden besitzt, da ja dem Gelde 
alles untertan ist.
\end{quote}
\begin{flushright}
\textsl{Thomas von Aquin (1225-1274)}
\end{flushright}
Daraus folgt: Das äußere Environment bestimmt das innere! Das ist bei Schachtelungen zu beachten. In manchen Fällen widersprechen sich auch die Einstellungen der verschachtelten Environments, dies ist dann im Einzelnen zu analysieren.

\subsection{Ausrichtungen}

Die Ausrichtung des Textes sollte man meist \LaTeX\ und der verwendeten Dokumentklasse überlassen. Will man jedoch einen Text besonders hervorheben, kann man dies mit dem \texttt{center}-Environment für zentrierten Text:
\index{Zentrierter Text}
\begin{center}
\textsc{Zentrierter Text}
\end{center}

\noindent Dem \texttt{flushleft}-Environment für linksorientierten Text (default):

\index{Linkbündig}
\begin{flushleft}
\textsc{Links ausgerichteter Text}
\end{flushleft}

\noindent Und dem \texttt{flushright}-Environment für rechtsorientierten Text:

\index{Rechtsbündig}
\begin{flushright}
\textsc{Rechts ausgerichteter Text}
\end{flushright}

\subsection{Abbildungen}\index{Abbildungen}

Abbildungen, die auch im Abbildungsverzeichnis auftauchen sollen, formatiert man mit dem \texttt{figure}-Environment. Eine Abbildung wird mit diesem Environment an eine Stelle gesetzt, die dem Seitenbild nach gewissen Regeln "`förderlich"' ist. Da Sie nie genau wissen, wohin \LaTeX\ Ihre Abbildung verschiebt, ist es notwendig, diese nummerieren zu lassen (was automatisch geschieht) und diese Nummer im Text als Referenz zu verwenden. Die im Folgenden dargestellten Abbildungen haben keine Verweisangaben, da dies in Abschnitt \ref{sect:verweise} im Detail dargestellt wird. 

\index{figure-Umgebung}
\begin{figure}[h]
\centering
{\fontsize{48}{50} \staveXI}
\caption{Eine Beispiel Abbildung.}
\end{figure}

\begin{verbatim}
\begin{figure}[h]
\centering
{\fontsize{48}{50} \staveXI}
\caption{Eine Beispiel Abbildung.}
\end{figure}
\end{verbatim}
Die eckigen Klammern hinter dem \texttt{begin}-Befehl stellen Parameter zum gewählten Environment dar. Das \texttt{figure}-Environment bietet für den Parameter die Werte:
\begin{description}
\item[h] Für die Stelle, an der die Abbildung definiert wurde.
\item[b] Für den unteren Rand (bottom) einer Seite. 
\item[t] Für den oberen Rand (top) einer Seite.
\item[p] Die Zeichnung soll auf einer eigenen Seite erscheinen.
\end{description}
Eine Abbildung benötigt immer eine gewisse Menge Platz. Sollte der Parameter \texttt{h} verwendet worden sein und an der Stelle, wo die Abbildung eingesetzt werden soll, nicht mehr genug Platz auf der Seite zur Verfügung stehen, wird sie trotz des \texttt{h} Parameters auf die nächste Seite verschoben und der Text, der eigentlich hinter der Abbildung auftauchen sollte, wird davor gesetzt.

\section{Andere Dokument-Klassen}

\subsection{\texttt{amsbook}-Klasse}\index{amsbook-Klasse}
TODO

\subsection{\texttt{memoir}-Klasse}\index{memoir-Klasse}
TODO

\subsection{\texttt{tufte-book}-Klasse}\index{tufte-book-Klasse}
TODO

\subsection{\texttt{tufte-handout}-Klasse}\index{tufte-handout-Klasse}
TODO


\section{Lizenzen}\index{Lizenz}

Diverse professionelle Verlage haben eigene Dokumentklassen erzeugt und stellen diese kostenlos zum Download zur Verfügung. Dies soll in erster Linie den Autoren, die ein Sachbuch für diesen Verlag schreiben, als Erleichterung dienen -- und natürlich dem Verlag, der dann später nicht das gesamte Dokument umformatieren muss. 

Die Verlage stellen diese Klassen in aller Regel ohne weitere Lizenz-An\-ga\-ben zur Verfügung. Läge eine GNU Public License dabei oder irgendetwas anderes, wäre die Situation klar. Aber so begibt man sich erstmal in eine rechtliche Grauzone. Meiner Meinung nach -- und ich bin kein Jurist, also ist dies ohne Gewähr zu verstehen -- legen die Verlage auf die Form (sprich die Dokumentklasse) keinen ernsthaften Wert. Für sie ist der Inhalt von Relevanz und darauf baut ihr Geschäftsmodell auf, nicht die Form der Veröffentlichung. Daher liegt diesen Dokumentklassen vermutlich auch keine Lizenz bei. 

Ich gehe davon aus, dass man, solange man nicht kommerziell mit den Verlagsklassen arbeitet, diese ohne Weiteres verwenden kann. Man sollte allerdings jegliche Verlags-Logos und -Namen entfernen. Bei einer Abschlussarbeit bekommt man da vermutlich keinerlei Probleme.

Will man sich vollständig absichern, lohnt es sich, dem Verlag eine Mail zu schreiben und nachzufragen, ob man die Klasse für eine Abschlussarbeit verwenden darf. Der Verlag wird -- sofern er sich zu einer Antwort aufraffen kann -- dies vermutlich nicht ablehnen, allerdings wie oben beschrieben, darauf bestehen, dass Logos und Verlagsname entfernt wird. 

Im Fall der American Mathematical Society gibt es die \texttt{amsbook}-Klasse, die den \LaTeX-Distributionen grundsätzlich beiliegt. Auf dieser liegt selbstverständlich keine proprietäre Lizenz und man kann sie auch für kommerzielle Projekte nutzen. Allerdings hat die AMS auch weitere Klassen, wie zum Beispiel die \texttt{gsm-l}-Klasse für die \textsl{Graduate Studies in Mathematics} Reihe. Hier muss man schon etwas vorsichtiger sein. Allerdings steht im Klassenfile der Hinweis, dass, falls die Klasse umbenannt wird, sie auch verändert werden darf, was einer Nutzung in anderem kommerziellen Zusammenhang entsprechen würde. 

Aber, um es noch einmal zu sagen: Ich bin kein Jurist, also sind alle Hinweise hier ohne Gewähr. Wenn Sie eine kommerzielle Klasse einsetzen möchten, machen Sie sich bitte selbst schlau. 


\section{Source Code und Algorithmen}

Es gibt verschiedene Möglichkeiten, wie Sie Source Code und Algorithmen darstellen können. Der einfachste Weg ist die im Default enthaltene \texttt{verbatim}-Umgebung zu verwenden. Diese wird auch in diesem Buch oft verwendet, um \LaTeX\ Befehle darzustellen.

\begin{verbatim}
Die verbatim-Umgebung lässt es zu, Text genauso in das Dokument
zu übernehmen, wie er geschrieben wurde. Daher wird in aller
Regel auch ein Monospace-Font verwendet. 
\end{verbatim}
manche Dokumentklassen nutzen keinen Monospace Font für den \texttt{verbatim}-Bereich. Ob Ihre Source-Code und Algorithmen dann noch klar dargestellt werden, müssen Sie entscheiden. 

\subsection{\texttt{lstlisting}-Paket}

Zur schöneren Darstellung von Source-Code kann das \texttt{lstlisting}-Paket verwendet werden. In Abbildung \ref{fig:java} dargestellter Java Code ist mit diesem Paket formatiert.

\begin{figure}[h]
\begin{lstlisting}
public class AppFX extends javafx.application.Application 
      implements EventHandler<WindowEvent> {

    public static void main(String[] args) throws Exception {
        launch(args);
    }
}
\end{lstlisting}
\caption{Beispiel Java Code}
\label{fig:java}
\end{figure}

Es werden vom \texttt{lstlisting}-Paket die folgenden Sprachen unterstützt:

ABAP, ACSL, Ada, Algol, Ant, Assembler, Awk, bash, Basic, C, C++, Caml, Clean, Cobol, Comal, csh, Delphi, Eiffel, Elan, erlang, Euphoria, Fortran, GCL, Gnuplot, Haskell, HTML, IDL, inform, Java, JVMIS, ksh, Lisp, Logo, make, Mathematika, Matlab, Mercury, MetaPost, Miranda, Mizar, ML, Modelica, Modula-2, MuPAD, NASTRAN, Oberon-2, OCL, Octave, Oz, Pascal, Perl, PHP, PL/1, Plasm, POV, Prolog, Promela, Python, R, Reduce, Rexx, RSL, Ruby, S, SAS, sh, SHELXL, Simula, SQL, tcl, TeX, VBScript, Verilog, VHDL, VRML, XML, XSLT.


\chapter{Zitieren und Literatur}

Im Besonderen für wissenschaftliche Werke und Abschlussarbeiten ist der korrekte Umgang mit Literaturverweisen eine absolute Notwendigkeit. \LaTeX\ unterstützt den Anwender hier in zwei grundsätzlichen Arten. Zunächst in Form einer einfachen Literaturliste, auf die vom Text aus verwiesen werden kann. Und in der etwas komplexeren Art mit dem BibTeX Tool. Letzteres ist zwar etwas komplizierter, nimmt einem aber so viel Arbeit ab, dass dies der Weg zum Literaturverzeichnis sein sollte. 

\section{Einfaches Verzeichnis}

Der Vollständigkeit halber wird hier auch das einfache Verzeichnis erwähnt. Bei wenigen Verweisen kann dies auch ausreichen. Die Syntax zur Definition eines Literaturverweises ist die folgende:
\begin{verbatim}
\bibitem{interne-referenz}Autor(en):
           \emph{Titel der Veröffentlichung}. Verlag,
           Stadt -- ggf. Land, Jahr.
\end{verbatim}
Beispiel:
\begin{verbatim}
\bibitem{Brieskorn1}
  Egbert Brieskorn,
  \emph{Lineare Algebra und analytische Geometrie I}.
  Vieweg, Wiebaden; Braunschweig,
  1. Auflage, Nachdruck,
  1983/1985.
\end{verbatim}

So erzeugte Literatureinträge müssen vom Anwender selbst in die korrekte Reihenfolge gebracht und an der richtigen Stelle ins Dokument eingefügt werden. Dies sollte auf jeden Fall über eine eigenständige \LaTeX-Datei erfolgen, z.B. \texttt{literatur.tex}. Ein vollständiges Literaturverzeichnis sieht so aus:
\begin{verbatim}
\begin{thebibliography}{123}

\bibitem{Brieskorn1}
  Egbert Brieskorn,
  \emph{Lineare Algebra und analytische Geometrie I}.
  Vieweg, Wiebaden; Braunschweig,
  1. Auflage, Nachdruck,
  1983/1985.

\end{thebibliography}
\end{verbatim}
Das \texttt{thebibliography}-Environment stellt die Literaturangaben auf einer eigenen Seite mit der Überschrift "`Literaturverzeichnis"' dar. Zu beachten ist, dass der Parameter 123 nichts weiter bedeutet, als dass möglicherweise dreistellige Literaturangaben entstehen könnten (das sind 1000 Referenzen). Es können durchaus weniger sein, das spielt keine Rolle. 

Eine Referenz im Text auf diese Literaturangabe wird über den Befehl
\begin{verbatim}
\cite{Brieskorn1}
\end{verbatim}
realisiert. Üblicherweise ist dies dann eine Angabe der Form "`[1]"' (ohne Anführungsstriche). Die Literaturangaben werden nicht sortiert, das bedeutet, sie werden in der Reihenfolge angezeigt, wie sie im literatur.tex File enthalten sind. 

\section{Das BibTeX Tool}

Eine sehr viel elegantere Methode zur Verwaltung der Literaturangaben bietet das BibTeX Tool, sie werden in einer quasi Datenbank abgelegt. Der Begriff "`Datenbank"' ist an dieser Stelle etwas hoch gegriffen, die Literaturangaben werden lediglich in einem File gesammelt, unabhängig davon, ob sie jemals verwendet werden, oder nicht. Das erspart Arbeit (auch wenn es nicht so klingt), denn der Anwender braucht sich keine Gedanken darüber zu machen, ob er je die Literaturangabe benötigt. Macht er keine Referenz darauf, taucht sie nicht im Literaturverzeichnis auf. Zusätzlich werden die Literaturangaben automatisch nach dem Nachnamen des Autors alphabetisch sortiert. 

\subsection{Aufbau der Datenbank}

BibTeX hat ein eigenes Format, wie Literatur Referenzen gespeichert werden. An einem Beispiel wollen wir uns das näher ansehen:
\begin{verbatim}
@BOOK{Brieskorn1983, 
	title="{Lineare Algebra und Analytische Geometrie I: 
	Noten zu einer Vorlesung mit historischen Anmerkungen 
	von Erhard Scholz (German Edition)}",
	author={Egbert Brieskorn},
	publisher={Vieweg+Teubner Verlag},
	year={1983},
	month={1},
	edition={1983},
	isbn={9783528085612},
	url={http://amazon.com/o/ASIN/3528085614/},
	price={$49.99},
	totalpages={636},
	timestamp={2014.01.30},
}
\end{verbatim}
Das sieht in keiner Weise einfacher aus, als die Angabe oben. Aber das täuscht, denn letztlich brauchen wir nur diejenigen Informationen, die auch wirklich angezeigt werden. Also diese:
\begin{verbatim}
@BOOK{Brieskorn1983, 
	title="{Lineare Algebra und Analytische Geometrie I: 
	Noten zu einer Vorlesung mit historischen Anmerkungen 
	von Erhard Scholz (German Edition)}",
	author={Egbert Brieskorn},
	publisher={Vieweg+Teubner Verlag},
	year={1983},
	month={1},
	edition={1983},
}
\end{verbatim}
Unabhängig davon ist das BibTeX Format aber allgegenwärtig, sodass viele Bibliotheken und Journale bereits BibTeX Informationen zu ihren Veröffentlichungen anbieten. Diese müssen dann einfach nur kopiert und in die BibTeX Datenbank hinein kopiert werden. 

Das oben angegebene Buch von Egbert Brieskorn würde dann als BibTeX Eintrag in der folgenden Art zitiert: \cite{Brieskorn1983a}, der zweite Band ist \cite{Brieskorn1983b}

In Ihrem \TeX-Dokument müssen dann die Zeilen
\begin{verbatim}
\bibliographystyle{amsalpha}
\bibliography{literature}
\end{verbatim}
an der Stelle erscheinen, wo das Literaturverzeichnis auftauchen soll, also meistens am Ende. 

Wir verwenden hier den \texttt{amsalpha}-Style für die Literaturreferenzen. Das führt zu Referenzen der form \cite{Raymond:2014uha}. Der \texttt{plain}-Style würde zu Referenzen der Form [1], [2], usw. führen.







\chapter{Tabellen}





\chapter{Installation von Paketen}\label{chap:instpack}



\chapter{Formelsatz}\label{chap:formel}

Eine der wichtigsten Anwendungen von \LaTeX\ ist der Formelsatz. Es ist zum quasi Standard geworden, sobald Formeln in einer Veröffentlichung auftauchen. Ich hoffe, in den vorherigen Kapiteln deutlich gemacht zu haben, dass es sich auch dann lohnt \LaTeX\ einzusetzen, wenn man keine Formeln verwendet. Aber sollte man es tun, führt praktisch kein Weg an \LaTeX\ vorbei.

Mathematische Formeln werden in verschiedenen Situationen eingesetzt. Im Fließtext (inline) und zum anderen als zentrierte Formel in einem eigenen Absatz (Display-Style). Alles, was über den Formelsatz im Display-Style gesagt werden kann, trifft auch für den inline Satz zu. Allerdings werden die Formeln im inline Satz meist flacher dargestellt, was dazu führt, dass hoch- oder tiefgestellte Informationen gegebenenfalls an anderen Stellen auftauchen, doch dazu später mehr.

Der Display-Style von Formeln stellt diese -- wie gesagt -- zentriert in einem eigenen Absatz dar:

\begin{equation}
\mathcal{F}(f)(t) = \frac{1}{(2\pi)^{\frac{n}{2}}} 
	\int_{\mathbb{R}^n} f(x)e^{-itx} dx
\end{equation}

Der Display-Style wird eingesetzt, wenn Code im \texttt{equation}-Environment geschrieben wird:
\begin{verbatim}
\begin{equation}
\mathcal{F}(f)(t) = \frac{1}{(2\pi)^{\frac{n}{2}}} 
	\int_{\mathbb{R}^n} f(x)e^{-itx} dx
\end{equation}
\end{verbatim}
Das \texttt{equation}-Environment ist die gebräuchlichste Umgebung für eine Formel. Sie beinhaltet auch eine Nummerierung, die per Default an den rechten Seitenrand gesetzt wird. Will man auf eine bestimmte Formel verweisen, sollte diese mit einem Label versehen werden und über den \texttt{ref}-Befehl referenziert werden. Dies wird in Abschnitt \ref{sect:verweise} noch genauer beschrieben.

Will man keine Nummerierung der Formel, kann das \texttt{equation*}-En\-vi\-ron\-ment verwendet werden:
\begin{verbatim}
\begin{equation*}
\mathcal{F}(f)(t) = \frac{1}{(2\pi)^{\frac{n}{2}}} 
	\int_{\mathbb{R}^n} f(x)e^{-itx} dx
\end{equation*}
\end{verbatim}
ergibt
\begin{equation*}
\mathcal{F}(f)(t) = \frac{1}{(2\pi)^{\frac{n}{2}}} 
	\int_{\mathbb{R}^n} f(x)e^{-itx} dx
\end{equation*}
Nicht nummerierte Formel können auch mit der Abkürzung
\begin{verbatim}
\[
\mathcal{F}(f)(t) = \frac{1}{(2\pi)^{\frac{n}{2}}} 
	\int_{\mathbb{R}^n} f(x)e^{-itx} dx
\]
\end{verbatim}
geschrieben werden. Verwenden Sie nicht den von \TeX\ bekannten "`\$\$"' (doppel-Dollar) Display-Satz. Dieser ist im \LaTeX-Umfeld nicht mehr gebräuchlich und führt unter bestimmten Umständen zu einem falschen Abstands- und Umbruchverhalten.

\section{Griechische Zeichen}

Die in Tabelle \ref{tab:greek} dargestellten griechischen Zeichen stehen nur im Mathematik Modus zur Verfügung -- abgesehen natürlich von den Zeichen, die durch normale lateinische Zeichen dargestellt werden. Daher sind sie sinnvoll nur dann zu verwenden, wenn sie in Formeln vorkommen. 

Sollen Absätze oder ganze Texte in griechisch formatiert werden, müssen Pakete verwendet werden, bzw. falls der gesamte Text in griechisch gesetzt werden soll, reicht es das Font-Encoding anzupassen.

\begin{table}[t]
\centering
\begin{tabular}{c|l|c|l|l}
\hline
\textbf{Kleiner Buchstabe} & \textbf{\LaTeX-Code} & \textbf{Großbuchstabe} & \textbf{\LaTeX-Code} & \textbf{Bezeichnung} \\
\hline
$\alpha $ & $\backslash$alpha & $A $ & A & Alpha \\
$\beta $ & $\backslash$beta & $B $ & B & Beta \\
$\gamma $ & $\backslash$gamma & $\Gamma $ & $\backslash$Gamma & Gamma \\
$\delta $ & $\backslash$delta & $\Delta $ & $\backslash$Delta & Delta \\
$\epsilon$ & $\backslash$epsilon  & $E $ & E & Epsilon \\
$\zeta $ & $\backslash$zeta & $Z $ & Z & Zeta \\
$\eta $ & $\backslash$eta & $H $ & H &  Eta\\
$\theta $ & $\backslash$theta & $\Theta $ & $\backslash$Theta & Theta \\
$\iota $ & $\backslash$iota & $I $ & I & Iota \\
$\kappa $ & $\backslash$kappa & $K $ & K & Kappa \\
$\lambda $ & $\backslash$lambda & $\Lambda $ & $\backslash$Lambda & Lambda \\
$\mu $ & $\backslash$mu & $M $ & M & Mu \\
$\nu $ & $\backslash$nu & $N $ & N & Nu \\
$\xi $ & $\backslash$xi & $\Xi $ & $\backslash$Xi &  Xi \\
$o $ & o & $O $ & O & Omicron \\
$\pi $ & $\backslash$pi & $\Pi $ & $\backslash$Pi & Pi \\
$\rho $ & $\backslash$rho & $P $ & P & Rho \\
$\sigma $ & $\backslash$sigma & $\Sigma $ & $\backslash$Sigma & Sigma  \\
$\tau $ & $\backslash$tau & $T $ & T & Tau \\
$\upsilon $ & $\backslash$upsilon & $\Upsilon $ & $\backslash$Upsilon & Ypsilon \\
$\phi $ & $\backslash$phi & $\Phi $ & $\backslash$Phi & Phi \\
$\chi $ & $\backslash$chi & $X $ & X & Chi \\
$\psi $ & $\backslash$psi & $\Psi $ & $\backslash$Psi & Psi \\
$\omega $ & $\backslash$omega & $\Omega $ & $\backslash$Omega & Omega \\
\hline
\end{tabular}
\caption{Griechische Buchstaben}
\label{tab:greek}
\end{table}

\section{Exponenten und Indizes}

Exponenten werden mit dem $\hat{ }$\ -Zeichen hochgestellt. Das "`hoch"'-Zeichen gilt immer nur für das darauf folgende Zeichen. Sollen mehrere Zeichen in den Exponent, \emph{müssen} Sie mit Blockklammern arbeiten.
\begin{verbatim}
e^-itx
\end{verbatim}
ergibt
\begin{equation}
e^-itx
\end{equation}
während
\begin{verbatim}
e^{-itx}
\end{verbatim}
zu 
\begin{equation}
e^{-itx}
\end{equation}
wird. Das selbe gilt für tiefgestellte Indizes. Diese werden mit dem \_-Zeichen erzeugt:
\begin{verbatim}
A_{i,j}
\end{verbatim}
ergibt
\begin{equation}
A_{i,j}
\end{equation}
während 
\begin{verbatim}
A_i,j
\end{verbatim}
zu
\begin{equation}
A_i,j
\end{equation}



\section{Matrizen}


\section{Mehrzeilige Formeln}








\chapter{Feintuning}


\section{Verweise}\label{sect:verweise}

TODO label und ref.

\section{Index}

TODO


\section{Aufzählungen}

\subsection{Enumerierungen}

Es gibt verschiedene Aufzählungsarten in \LaTeX. Beginnen wir mit der einfachen "`Aufzählung"'. Diese wird durch das "`enumerate"'-Environment erreicht.\index{enumerate-Umgebung}
\begin{verbatim}
\begin{enumerate}
\item erster
\item zweiter ...
\item dritter ...
\end{enumerate}
\end{verbatim}
wird in der Ausgabe zu:
\begin{enumerate}
\item erster
\item zweiter zweiter zweiter zweiter zweiter zweiter zweiter zweiter zweiter zweiter zweiter zweiter zweiter zweiter zweiter zweiter zweiter
\item dritter dritter dritter dritter dritter dritter dritter dritter dritter dritter dritter dritter dritter dritter dritter dritter dritter dritter dritter 
\end{enumerate}

Wie man sehen kann, werden auch längere Angaben so eingerückt, dass die Nummerierung klar zu sehen ist. Der \texttt{item}-Befehl hat ein Argument, das in eckigen Klammern angegeben werden kann. Es ersetzt die Nummer und dies führt dazu, dass kein Erhöhen der Nummer stattfindet. Gibt man also an der dritten Stelle im enumerate-Envrionment den Befehl
\begin{verbatim}
\item[-] dritter
\end{verbatim}
an, so wird keine Nummer angegeben, sondern der "`-"' (Dash) und ein vierter Eintrag wird die Nummer drei bekommen.

Am besten einfach mal etwas damit herumspielen, dann ergeben sich die Besonderheiten schon von selbst.

\subsection{Bullet-Liste}

Die nächste Möglichkeit der Aufzählungen ist nicht nummeriert, hierfür wird das "`itemize"'-Environment verwendet. Das Aufzählungszeichen ist per Default der "`\textbullet"' (\texttt{textbullet}).\index{itemize-Umgebung}
\begin{verbatim}
\begin{itemize}
\item erster
\item[\textasteriskcentered] zweiter
\item dritter
\end{itemize}
\end{verbatim}

\begin{itemize}
\item erster
\item[\textasteriskcentered] zweiter
\item dritter
\end{itemize}
Wie auch bei den Aufzählungen kann mit den eckigen Klammern das Aufzählungszeichen verändert werden. 

\subsection{Dingbat Listen}

Das \texttt{pifont} Paket bietet ein \texttt{dinglist}-Environment, indem das Aufzählungszeichen als Parameter übergeben werden kann. Hier z.B. die Nummer 229 -- ein Pfeil.

\begin{verbatim}
\begin{dinglist}{229}
  \item foo
  \item bar
\end{dinglist}
\end{verbatim}

\begin{dinglist}{229}
  \item foo
  \item bar
\end{dinglist}

\subsection{Sonstiges}

\index{description-Umgebung}
TODO

\section{\texttt{include} und \texttt{import}}\label{sect:import}

TODO

\backmatter

\bibliographystyle{amsalpha}
\bibliography{literature}

\printindex

\end{document}
