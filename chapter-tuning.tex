
\chapter{Feintuning}


\section{Verweise}\label{sect:verweise}

TODO label und ref.

\section{Index}

TODO

\section{Text-Formatierung}

\LaTeX\ formatiert Text grundsätzlich im Blocksatz. Um die korrekte Zeilenlänge zu erreichen, wird in erster Linie die Silbentrennung (engl. hyphenation) verwendet. Sind die Wortzwischenräume trotz Silbentrennung zu groß, kommen weitere Formatierungshilfen zum Einsatz, deren Verhalten man anhand diverser Parameter modifizieren kann.

\subsection{Blocksatz und seine Parameter}

TODO

\subsection{Flattersatz}

\begin{flushleft}
In einigen Dokumenten ist der Blocksatz unerwünscht. Oft ist auch gerade die Silbentrennung nicht gewollt. Verlage bevorzugen Texte im linksbündigen Flattersatz ohne Silbentrennung bei einer 12pt Courier Schrift mit 30 Zeilen und maximal 60 Zeichen pro Zeile. Möchte man dies erfüllen, steht man vor dem Problem, dass \LaTeX\ großen Wert darauf legt, dass Absätze rechteckig sind und hierfür Silbentrennung und dynamische Wortabstände einsetzt, um dies zu erreichen. 

Dieses Verhalten kann man ausschalten, wenn man (siehe Abschnitt \ref{sect:ausrichtung}) das \texttt{flushleft}-Environment verwendet. Dies ist allerdings meist nur praktikabel, wenn man nicht mehr als einen Absatz so darstellen möchte. Will man ein gesamtes Dokument auf Flatterstatz umstellen, so gibt es den Befehl \texttt{flushleft}, der dies für jeglichen folgenden Text einstellt. 

\end{flushleft}


\section{Aufzählungen}

\subsection{Enumerierungen}

Es gibt verschiedene Aufzählungsarten in \LaTeX. Beginnen wir mit der einfachen "`Aufzählung"'. Diese wird durch das "`enumerate"'-Environment erreicht.\index{enumerate-Umgebung}
\begin{verbatim}
\begin{enumerate}
\item erster
\item zweiter ...
\item dritter ...
\end{enumerate}
\end{verbatim}
wird in der Ausgabe zu:
\begin{enumerate}
\item erster
\item zweiter zweiter zweiter zweiter zweiter zweiter zweiter zweiter zweiter zweiter zweiter zweiter zweiter zweiter zweiter zweiter zweiter
\item dritter dritter dritter dritter dritter dritter dritter dritter dritter dritter dritter dritter dritter dritter dritter dritter dritter dritter dritter 
\end{enumerate}

Wie man sehen kann, werden auch längere Angaben so eingerückt, dass die Nummerierung klar zu sehen ist. Der \texttt{item}-Befehl hat ein Argument, das in eckigen Klammern angegeben werden kann. Es ersetzt die Nummer und dies führt dazu, dass kein Erhöhen der Nummer stattfindet. Gibt man also an der dritten Stelle im enumerate-Envrionment den Befehl
\begin{verbatim}
\item[-] dritter
\end{verbatim}
an, so wird keine Nummer angegeben, sondern der "`-"' (Dash) und ein vierter Eintrag wird die Nummer drei bekommen.

Am besten einfach mal etwas damit herumspielen, dann ergeben sich die Besonderheiten schon von selbst.

\subsection{Bullet-Liste}

Die nächste Möglichkeit der Aufzählungen ist nicht nummeriert, hierfür wird das "`itemize"'-Environment verwendet. Das Aufzählungszeichen ist per Default der "`\textbullet"' (\texttt{textbullet}).\index{itemize-Umgebung}
\begin{verbatim}
\begin{itemize}
\item erster
\item[\textasteriskcentered] zweiter
\item dritter
\end{itemize}
\end{verbatim}

\begin{itemize}
\item erster
\item[\textasteriskcentered] zweiter
\item dritter
\end{itemize}
Wie auch bei den Aufzählungen kann mit den eckigen Klammern das Aufzählungszeichen verändert werden. 

\subsection{Dingbat Listen}

Das \texttt{pifont} Paket bietet ein \texttt{dinglist}-Environment, indem das Aufzählungszeichen als Parameter übergeben werden kann. Hier z.B. die Nummer 229 -- ein Pfeil.

\begin{verbatim}
\begin{dinglist}{229}
  \item foo
  \item bar
\end{dinglist}
\end{verbatim}

\begin{dinglist}{229}
  \item foo
  \item bar
\end{dinglist}

\subsection{Sonstiges}

\index{description-Umgebung}
TODO

\section{\texttt{include} und \texttt{import}}\label{sect:import}

TODO